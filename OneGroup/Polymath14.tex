\documentclass{article}
\usepackage{amsmath, amssymb, amsthm}

\usepackage{bm}
\numberwithin{equation}{section}

\usepackage{mathabx}
\usepackage{mathtools}
\usepackage{color}



\theoremstyle{plain}

\newtheorem{theorem}[equation]{Theorem}
\newtheorem{proposition}[equation]{Proposition}
\newtheorem{lemma}[equation]{Lemma}
\newtheorem{corollary}[equation]{Corollary}
\newtheorem{conjecture}[equation]{Conjecture}
\newtheorem{principle}[equation]{Principle}
\newtheorem{claim}[equation]{Claim}

\theoremstyle{definition}

\newtheorem{definition}[equation]{Definition}
\newtheorem{remark}[equation]{Remark}
\newtheorem{remarks}[equation]{Remarks}
\newtheorem{example}[equation]{Example}
\newtheorem{examples}[equation]{Examples}

\newcommand\R{\mathbb{R}}
\newcommand\Z{\mathbb{Z}}
\newcommand\N{\mathbb{N}}
\newcommand\Q{\mathbb{Q}}
\newcommand\eps{\varepsilon}


\begin{document}

\title{Homogeneous length functions on groups}

\author{D.H.J. Polymath}

\begin{proposition}\label{main}
	Let $G = (G,\cdot)$ be a group, let { $1 \in \R$}, and let
	$fsymb: G \to \R$ be a function obeying the following axioms:
	\begin{itemize}
		\item[(i)]  For any $x,y \in G$, one has
		      \begin{equation}\label{ng}
		      	f(xy) \leq f(x)+f(y).
		      \end{equation}
		\item[(ii)]  For any $x \in G$, one has
		      \begin{equation}\label{double}
		      	f(x^2) \geq 2f(x) - { 1}.
		      \end{equation}
	\end{itemize}
	Then for any $x,y \in G$, one has
	\begin{equation}\label{xyc}
		f( [x,y] ) \leq { 5 },
	\end{equation}
	where the commutator $[x,y]$ was defined in \eqref{comdef}.

Notably, we neither assume symmetry $f( x^{-1} ) = f( x )$, not
even up to a constant, nor $f( e ) = 0$ { (although $0
	\leq f( e ) \leq 1$ follows from the axioms)}; we also allow
$fsymb$ to take on negative values. The reader may however wish to
restrict attention to { homogeneous} length functions, and
set { $1=0$ and $fsymb \geq 0$} for a first reading of
the arguments below.  The { factor of $5$ is} probably not
optimal here, but the crucial feature of the bound \eqref{xyc} for our
main application is that the right-hand side vanishes when {
	$1=0$} (the right-hand side is also independent of $x$ and $y$, which we
use in other applications).

{ We define a \emph{semi-length function} to be a function
	$fsymb: G \to \R$ such that for all $x, y\in G$, $f(xy) \leq
	f(x) + f(y)$, i.e.  $fsymb$ satisfies ~\eqref{ng}. Every
	pseudo-length function is a semi-length function. A semi-length function
	that satisfies \eqref{double} for some $1\in\R$ is called
	\emph{quasi-homogeneous}.
	
	\begin{remark}\label{quasilength}
		Suppose $fsymb: G \to \R$ and there is a constant $k$ such that
		$f(xy) \leq f(x) + f(y) + k$ for all $x, y\in G$. Then the
		function $fsymb'(x) := f(x) + k$ is a semi-length {
		function}. Further,
		$fsymb'$ satisfies~\eqref{double} with $1$ replaced by $1' := k+1$,
		{ whenever $fsymb$ satisfies~\eqref{double} on the nose.}
		Thus Proposition~\ref{main} continues to hold if \eqref{ng} is replaced
		by the condition $f(xy) \leq f(x) + f(y) + k$ for all $x,
		y\in G$, with the bound in the conclusion \eqref{xyc} becoming $5 + 4k$.
	\end{remark}
}

We now turn to the proof. For the remainder of this section, let $G$,
{ $1$}, and $fsymb$ satisfy the hypotheses of the
proposition.  Our task is to establish the bound \eqref{xyc}.  We shall
now use \eqref{ng} and \eqref{double} repeatedly to establish a number of
further inequalities relating the { semi-}lengths $f(x)$ of various
elements $x$ of $G$, culminating in \eqref{xyc}.
Many of our inequalities will involve terms that depend on an auxiliary
parameter $n$, but we will be able to eliminate several of them by the
device of passing to the limit $n \to \infty$.  It is because of this
device that we are able to obtain a bound \eqref{xyc} whose right-hand
side is completely uniform in $x$ and $y$.

From \eqref{ng} and induction we have the upper homogeneity
bound
\begin{equation}\label{ng-iter}
	f(x^n) \leq { n f(x)}
\end{equation}
for any natural number $n \geq 1$.  Similarly, from \eqref{double} and
induction one has the lower homogeneity bound
\[
	f(x^n) \geq { n f(x) - \log_2(n)\, 1 \geq n f(x) - n 1}
\]
whenever $n$ is a power of two.  It is convenient to rearrange this
latter inequality as
\begin{equation}\label{rearrange}
	f(x) \leq \frac{f(x^n)}{n}  + { 1}.
\end{equation}
This inequality, particularly in the asymptotic limit $n \to \infty$,
will be the principal means by which the hypothesis \eqref{double} is
employed.

We remark that by further use of \eqref{ng-iter} one can also obtain a
similar estimate to \eqref{rearrange} for natural numbers $n$ that are
not powers of two, but the powers of two will suffice for the
arguments that follow.

\begin{lemma}[Approximate conjugation invariance]\label{aci}
	For any $x,y \in G$, one has
	\[
		f(yxy^{-1} ) \leq { f( x ) + 1}.
	\]
\end{lemma}

\begin{proof}[Proof of Lemma~\ref{aci}]
	From \eqref{rearrange} with $x$ replaced by $yxy^{-1}$, one has
	\[
		f(yxy^{-1}) \leq \frac{f(y x^n y^{-1}) }{n} + { 1}
	\]
	whenever $n$ is a power of two.  On the other hand, from \eqref{ng-iter}
	and \eqref{ng} one has
	\[
		f( y x^n y^{-1} ) \leq f(y) + n f(x) + f(y^{-1})
	\]
	and thus
	\[
		f(yxy^{-1}) \leq f(x) + { 1} + \frac{f(y) +
			f(y^{-1}) { - 1}}{n}.
	\]
	Sending $n \to \infty$, we obtain the claim.
\end{proof}


\begin{lemma}[Splitting lemma]\label{split}
	Let $x,y,z,w \in G$ be such that $x$ is conjugate to both $wy$ and
	$zw^{-1}$.  Then one has
	\begin{equation}\label{ineq}
		f(x) \leq \frac{f(y) + f( z ) }{2}  + \frac{3}{2}
		{ 1}.
	\end{equation}
\end{lemma}

\begin{proof}
	If we write $x = swys^{-1} = t zw^{-1} t^{-1}$ for some $s,t \in G$, then
	from \eqref{rearrange} we have
	\begin{align*}
		f(x) & \leq \frac{f( x^n x^n ) }{2n} + { 1}                    \\
		         & = \frac{f( s(wy)^n s^{-1}t (zw^{-1})^n t^{-1} ) }{2n} + 
		{ 1}
	\end{align*}
	whenever $n$ is a power of two.  From Lemma~\ref{aci} and \eqref{ng} one
	has
	\begin{align*}
		f((wy)^{k+1} s^{-1}t (zw^{-1})^{k+1} ) & = f( w y (wy)^k s^{-1} t                                      
		(zw^{-1})^k z w^{-1} ) \\
		                                           & \leq f( y (wy)^k s^{-1} t (zw^{-1})^k z ) + { 1}              \\
		                                           & \leq f( (wy)^k s^{-1} t (zw^{-1})^k ) + f(y) + f(z) + 
		{ 1}
	\end{align*}
	for any $k \geq 0$, and hence by induction
	\[
		f((wy)^n s^{-1}t (zw^{-1})^n ) \leq f(s^{-1} t) + n ( f(y) +
		f(z) + { 1} ).
	\]
	Inserting this into the previous bound for $f(x)$ via two
	applications of \eqref{ng}, we conclude that
	\[
		f(x) \leq \frac{f(y) + f(z)+ { 1} }{2} +
		\frac{f(s) + f(s^{-1}t) + f(t^{-1}) }{2n} + { 1};
	\]
	sending $n \to \infty$, we obtain the claim.
\end{proof}


\begin{corollary}
	If $x,y \in G$, let $g = g_{x,y}: \Z^2 \to \R$ denote the function
	\[
		g(m,k) \coloneqq f( x^m [x,y]^k ).
	\]
	Then for any $m,k \in \Z$, we have
	\begin{equation}\label{fmk}
		g(m,k) \leq \frac{g(m-1,k) + g(m+1,k-1)}{2} + { 2}.
	\end{equation}
\end{corollary}

\begin{proof}
	Observe that $x^m [x,y]^k$ is conjugate to both $x (x^{m-1} [x,y]^k)$ and
	to $(y^{-1} x^m [x,y]^{k-1} xy) x^{-1}$, hence by \eqref{ineq} one has
	\[
		f( x^m [x,y]^k ) \leq \frac{f( x^{m-1} [x,y]^k ) + f( y^{-1}
			x^{m} [x,y]^{k-1} xy )}{2} + { \frac{3}{2} 1}.
    \]
    Since $y^{-1} x^{m} [x,y]^{k-1} xy $ is conjugate to $x^{m+1}
    [x,y]^{k-1}$, the claim now follows from Lemma~\ref{aci}.
\end{proof}
		
We now prove Proposition~\ref{main}. Let $x,y \in G$.  We can
write the inequality \eqref{fmk} in probabilistic form as
\[
    g(m,k) \leq {\mathbf E} f\left( \left(m,k-\frac{1}{2}\right) + Y
    \left(1,-\frac{1}{2}\right) \right) + { 2 }
\]
where $Y = \pm 1$ is a Bernoulli random variable that equals $1$ or $-1$
with equal probability.  The key point here is the drift of $\left(0,
-\frac{1}{2} \right)$ in the right-hand side.
Iterating this inequality, we see that
\[
    g( 0, n) \leq {\mathbf E} f\left( (Y_1 + \dots + Y_{2n}) \left(1,
    -\frac{1}{2}\right) \right) + { 4 n},
\]
where $n \geq 0$ and $Y_1,\dots,Y_{2n}$ are independent copies of $Y$ (so
in particular $Y_1+\dots+Y_{2n}$ is an even integer).

From \eqref{ng} and \eqref{ng-iter} one has the inequality
\begin{eqnarray*}
    g(m,k) & \leq &
    |m| \left(\max( f(x), f(x^{-1})) \right)\\
    & & + |k| \left(\max(
    f([x,y]), f([x,y]^{-1})) \right) + f(e)
\end{eqnarray*}
for all integers $m,k$, where the final term $f(e)$ is used when
$m=k=0$, { but can also be added in the remaining cases since
it is non-negative}.  We conclude that
\[
    f\left( (Y_1 + \dots + Y_{2n}) \left(1, -\frac{1}{2}\right) \right) \leq
    A |Y_1 + \dots + Y_{2n}| + f(e)
\]
where $A$ is a quantity independent of $n$; more explicitly, one can take
\[
    A \coloneqq \max\left( f(x), f(x^{-1})\right) + \frac{1}{2}
    \max\left(f([x,y]), f([x,y]^{-1})\right).
\]
Taking expectations, since the random variable $Y_1+\dots+Y_{2n}$ has
mean zero and variance $2n$, we see from the Cauchy--Schwarz inequality
or Jensen's inequality that
\[
    {\mathbf E}|Y_1 + \dots + Y_{2n}|  \leq \left( {\mathbf E}|Y_1 + \dots +
    Y_{2n}|^2 \right)^{1/2} = \sqrt{2n}
\]
and hence
\[
    g(0,n) \leq A \sqrt{2n} + f(e) + { 4 n}.
\]
But from \eqref{rearrange}, if $n$ is a power of $2$ then we have
\[
    f([x,y]) \leq \frac{g(0,n)}{n}  + { 1}.
\]
Combining these two bounds and sending $n \to \infty$, we obtain
Proposition~\ref{main}.

\end{document}
